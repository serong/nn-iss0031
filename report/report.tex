\documentclass[12pt,a4paper]{article}
\usepackage[utf8]{inputenc}
\usepackage{amsmath}
\usepackage{amsfonts}
\usepackage{amssymb}
\usepackage{graphicx}
\usepackage{listings}
\usepackage[section]{placeins}
\usepackage{gensymb}

% pagebreak after each section.
\usepackage{titlesec}
\newcommand\sectionbreak{\clearpage}



% ------------------ ALGO
\usepackage{algorithm}% http://ctan.org/pkg/algorithms
\usepackage{algpseudocode}% http://ctan.org/pkg/algorithmicx
\newcommand{\var}[1]{{\ttfamily#1}}% variable
% ------------------ ALGO

% Do while stuff. ------------------------
\algdef{SE}[DOWHILE]{Do}{doWhile}{\algorithmicdo}[1]{\algorithmicwhile\ #1}
% --------------------------------------

% Code color ------
\usepackage{color}
 
\definecolor{codegreen}{rgb}{0,0.6,0}
\definecolor{codegray}{rgb}{0.5,0.5,0.5}
\definecolor{codepurple}{rgb}{0.58,0,0.82}
\definecolor{backcolour}{rgb}{0.95,0.95,0.92}

\lstdefinestyle{mystyle}{
    backgroundcolor=\color{backcolour},   
    commentstyle=\color{codegreen},
    keywordstyle=\color{magenta},
    numberstyle=\tiny\color{codegray},
    stringstyle=\color{codepurple},
    basicstyle=\footnotesize,
    breakatwhitespace=false,         
    breaklines=true,                 
    captionpos=b,                    
    keepspaces=true,                 
    numbers=left,                    
    numbersep=5pt,                  
    showspaces=false,                
    showstringspaces=false,
    showtabs=false,                  
    tabsize=2
}
 
\lstset{style=mystyle}

% ----------------

\renewcommand\contentsname{}
\author{Serkan Ongan}
\begin{document}

\begin{titlepage}
	\centering
	{\scshape\Large Tallinn University of Technology \par}
	Faculty of Information Technology \par
	Department of Computer Control \par
	Chair of Automatic Control and Systems Analysis \par
	
	\vspace{4cm}
	
	Serkan Ongan 156395IASM \par
	{\large ISS0031 Modeling and Identification \par}
	{\Large\bfseries Practical Work 02\par}
	Fractional-order Modeling and Control \par
	
	\vspace{3cm}

	\begin{flushright}
	Supervisor  Aleksei \textsc{Tepljakov} \par
	\end{flushright}

	\vfill
	{Tallinn 2016 \par}
\end{titlepage}

\pagebreak

\section*{Table of Contents}

\tableofcontents

\pagebreak

\listoffigures

\pagebreak

\section{Control Problem}

\ \ \ \ Our task is to control the output $y(t)$ of a certain system. Following information about the system is known:

\begin{enumerate}
\item Input range u = [-1, 1].
\item Output range y = [0, 2pi].
\item Possible deadzone in the control and/or delay.
\end{enumerate}

We are also provided an identification dataset, along with corresponding validation data.

\subsection{Goal}

\ \ \ \ Goal is to identify a dynamic system and to design a fractional-order controller based on the model.

Control system specifications:

\begin{enumerate}
\item Time domain: Settling time $T_s \leq 20s$ for $y_r = [0,1]$, overshoot $v \leq 5 \%$ and error band $y_e = \pm 5\%$ of the set point $y_r$.
\item Frequency domain: Gain margin $G_m \geq 15dB$, phase margin $\varphi \geq 60 \degree$, also phase respone at critical frequency $w_c$ must be as flat as possible.
\end{enumerate}

\section{FOMCON}

\ \ \ \ To have a first look at the FOMCON toolbox. We can check if the following sample system is stable or not.

\begin{equation}
G = \frac{s + 1}{-s^1.5 - 1}
\end{equation}

Stability test shows us that the system is stable with q=1.5 order.

Also using the toolbox, we can simulate the system and/or check its bode diagram.

\begin{figure}[h]
	\caption{Impulse response}
	\includegraphics[scale=0.65]{images/000_1}
	\centering
\end{figure}

\FloatBarrier

\begin{figure}[h]
	\caption{Bode diagram.}
	\includegraphics[scale=0.4]{images/000_2}
	\centering
\end{figure}

\FloatBarrier

\ \ \ \ 

\section{Identification}

\ \ \ \ The first step is to load idsets.mat dataset, which we will use for identifying and control tasks.

\begin{verbatim}
proc_id_v1: identification
proc_id_v2: validation
\end{verbatim}

This is a SISO system and in the dataset, 1000 datapoints were collected with 0.05 time difference between samples.

The system inputs and outputs can be seen below.

\begin{figure}[h]
	\caption{System input and output.}
	\includegraphics[scale=0.35]{images/001}
	\centering
\end{figure}

\FloatBarrier

We can use FOTF Time-domain Identification Tool to start identifying our system. But we have delay and dead zone in the system, so the data needs to be trimmed. Our trim range is [2.25, 22.5]

\begin{figure}[h]
	\caption{Trimming data.}
	\includegraphics[scale=0.65]{images/002}
	\centering
\end{figure}

\FloatBarrier

\begin{figure}[h]
	\caption{Trimmed data.}
	\includegraphics[scale=0.65]{images/003}
	\centering
\end{figure}

\FloatBarrier

Next we start our initial identification starting with the following initial model.

\begin{equation}
G = \frac{1}{s + 1}
\end{equation}

\begin{figure}[h]
	\caption{Identification initial parameters.}
	\includegraphics[scale=0.65]{images/004}
	\centering
\end{figure}

\FloatBarrier

After initial identification we get the following model.

\begin{equation}
G = \frac{0.060674s^{0.043708}}{0.15414s^{0.77204}+0.11057s^{0.40657}}
\end{equation}

But dynamics in the residuals can be seen in after inspecting the result.


\begin{figure}[h]
	\caption{Residuals.}
	\includegraphics[scale=0.65]{images/005}
	\centering
\end{figure}

\FloatBarrier

\begin{figure}[h]
	\caption{Validation.}
	\includegraphics[scale=0.65]{images/006}
	\centering
\end{figure}

\FloatBarrier

After adding K (static gain) a new model with q=0.69 is calculated.

\begin{equation}
G = \frac{1}{14.554s^{0.69498}+1}
\end{equation}

And after checking residuals, we see that the dynamics disappeared in the model.

\begin{figure}[h]
	\caption{Residuals.}
	\includegraphics[scale=0.65]{images/007}
	\centering
\end{figure}

\FloatBarrier

\begin{figure}[h]
	\caption{Validation.}
	\includegraphics[scale=0.65]{images/008}
	\centering
\end{figure}

\FloatBarrier

After trying \textbf{different trimmed datasets}, no linear model could be reached. Often there were residual dynamics in the models.

And finally when we check the system with the validation data we get the following result. This is due the non-linearity in the system.

\begin{figure}[h]
	\caption{Id2 Validation.}
	\includegraphics[scale=0.65]{images/009}
	\centering
\end{figure}

\FloatBarrier

\section{Control Design}

\ \ \ \ First step to realize a control design is to open FPID Optimization Tool and load the provided $initial\_condition.mat$ configuration.

We also have the following Simulink model provided, where our initial conditions can be loaded into the simulation to check the control results.


\begin{figure}[h]
	\caption{Control model.}
	\includegraphics[scale=0.35]{images/100}
	\centering
\end{figure}

\FloatBarrier

And our initial conditions give us the following the result, which don't fit our control goals.

\begin{figure}[h]
	\caption{Initial control results.}
	\includegraphics[scale=0.35]{images/101}
	\centering
\end{figure}

\FloatBarrier

After optimization we get the following values for our control.

\begin{verbatim}
Gain margin [dB]: Inf
Phase margin [deg]: 105.77
\end{verbatim}

Both of these values fit our frequency domain goals, although as it can be seen later, there is a room for some improvement.

\begin{figure}[h]
	\caption{Open-loop frequency response comparison.}
	\includegraphics[scale=0.55]{images/102}
	\centering
\end{figure}

\FloatBarrier

\begin{figure}[h]
	\caption{Control optimization result.}
	\includegraphics[scale=0.55]{images/103}
	\centering
\end{figure}

\FloatBarrier

After loading the new conditions to the simulink and simulating the control, our Scope shows that we \textbf{partly} achieved our time domain goals as well.

The optimization achieved very small overshoot, with settling time smaller than 20s, and an error band of \%5.

\begin{figure}[h]
	\caption{Control results}
	\includegraphics[scale=0.35]{images/104}
	\centering
\end{figure}

\FloatBarrier

\begin{figure}[h]
	\caption{Control results.}
	\includegraphics[scale=0.35]{images/105}
	\centering
\end{figure}

\FloatBarrier

And if we enable "Critical Frequency and gain variation robustness" we can achieve a much better results in both time domain and frequency domain goals.

\begin{figure}[h]
	\caption{Control comparison.}
	\includegraphics[scale=0.55]{images/106}
	\centering
\end{figure}

\FloatBarrier

\begin{figure}[h]
	\caption{Open-loop frequency response comparison.}
	\includegraphics[scale=0.55]{images/107}
	\centering
\end{figure}

\FloatBarrier

\begin{figure}[h]
	\caption{Final control result.}
	\includegraphics[scale=0.35]{images/108}
	\centering
\end{figure}

\FloatBarrier

\ \ \ \ 

\end{document}